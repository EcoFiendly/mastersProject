The International Union for Conservation of Nature's (IUCN) Red List of Threatened Species\textsuperscript{TM} (hereafter 'Red List') is the primary source of extinction risk for life on Earth \parencite{iucn2021, rodrigues2006value, lamoreux2003value}. Although officially established in 1964 by The IUCN, the Red List dates back to a year prior in 1963 \parencite{scott1987red}. It started off as the International Red Data Books, which recorded threatened wildlife along with the threats they faced. Now, the IUCN states that the Red List is "Far more than a list of species and their status" \parencite{iucn2021}. 

Between now and its inception, the Red List had faced a number of criticisms. \textcite{master1991assessing} criticised the old extinction risk categories for lack of clarity and direction and discusses supplementary ranking methodologies. \textcite{mrosovsky1997iucn} questioned the Red List's credibility due to the witholding of key information behind decision making for some species. \textcite{possingham2002limits} emphasises that these lists should be used as a supplement rather than key factor in informing resource allocation (optimised by targetting species which can recover more using the same amount of resources, regardless of extinction category); reserve selection (look at all species occupying the selected area rather than select areas based on umbrella species); development or exploitation (listed/unlisted species can affect how resources are exploited [difficulty with explaination, ask for help]); reflect change in status of biodiversity (skewed representation of taxa results in accurate reflection of situation). \textcite{lamoreux2003value} largely agreed with \textcite*{possingham2002limits} but pointed out flaws in their reasoning and asserts that the Red List is an invaluable tool to aid conservation efforts. 

While acknowledging the limitations, \textcite{rodrigues2006value} defended the Red List and denounces some of the previous criticisms as misconceptions. The paper explains that the misconceptions, along with assumptions, prevented the Red List from achieving its potential. However, the conclusion converges with \textcite{possingham2002limits}, that the Red List should be used in conjunction with other tools to inform conservation. (can expand this paragraph with more details from the paper)

With the discussion surrounding the use of the Red List revolving around human interpretation, two projects have approached the data from a computational perspective. \textcite{gregg2020many} applied sentiment analysis on species common names and was able to conclude that some words were unfavourable as species names \parencite{gregg2020many}, preventing them from attracting sufficient attention and consequently, resources. Another research used the data and developed a software which incorporated generation length of individual species and status transition rates to simulate future extinctions \parencite{andermann2021iucn_sim}.

In 2000, the Red List was digitised \parencite{iucn2021} and the assessments now have to include supporting information for categorising the extinction risk for the species.
The availability of the information in plain text creates the opportunity for this project to approach the data from a computational perspective. Text-mining by machine reading was applied to try to answer the following research questions:
\begin{enumerate}
    \item What are the latent topics in the supporting texts.
    \item How do the topics vary across the different Red List categories.
    \item How do the topics vary across different systems (i.e. terrestrial, marine and freshwater)
    \item How do the topics vary across different biogeographic realms.
\end{enumerate}
