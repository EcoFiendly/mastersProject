The rapid development and growth of human civilisation and population is a constant threat to the world's biodiversity \parencite{joppa2016impact,cardinale2012biodiversity,hoffmann2010impact,duraiappah2005ecosystems, sala2000global}, with the Earth currently facing the sixth largest extinction \parencite{tilman2017future, barnosky2011has}. The International Union for Conservation of Nature's (IUCN) Red List of Threatened Species\textsuperscript{TM} (hereafter 'Red List') is one of the tools used to help combat biodiversity loss. The Red List is the primary source of extinction risk for life on Earth \parencite{iucn2021, rodrigues2006value, lamoreux2003value}. Although officially established in 1964 by The IUCN, the Red List dates back a year prior, to 1963 \parencite{scott1987red}. It started off as the International Red Data Books, which recorded threatened wildlife along with the threats they faced. Now, the IUCN states that the Red List is "Far more than a list of species and their status" \parencite{iucn2021}. 

Studies conducted with data from Red List assessments in the past have parameterised the data in order to perform their analyses. Red List Index (RLI) \parencite{iucn2009} and Evolutionary Distinct and Globally Endangered (EDGE) \parencite{isaac2007mammals} are two examples that have assigned weights to Red List categories (e.g. Least concern = 0, Near threatened = 1, etc) in order to incorporate them into mathematical models. Information that had been detailed in the assessments would be lost when studies select only the categories for use; this could have led to the criticisms faced by the Red List.

Between now and its inception, the Red List had faced a number of criticisms. \textcite{master1991assessing} criticised the old extinction risk categories for lack of clarity and direction and discusses supplementary ranking methodologies. \textcite{mrosovsky1997iucn} questioned the Red List's credibility due to the witholding of key information behind decision making for some species. \textcite{possingham2002limits} emphasised that these lists should be used as a supplement rather than as a key factor in informing conservation efforts, and made the following arguments. First, resource allocation can be optimised by targeting species which can recover more using the same amount of resources, regardless of extinction category. Second, reserve selection should look at all species occupying the selected area rather than select areas based on umbrella species. Third, development or exploitation where listed/unlisted species can affect how resources are exploited; listed species have been used to justify objection to development in certain areas; species that are under threat but unlisted for areas proposed for development would be affected when the development proceeds. Fourth, the list can be used to indicate change in the status of biodiversity, where skewed representation of taxa results in inaccurate reflection of the situation. \textcite{lamoreux2003value} largely agreed with \textcite{possingham2002limits} but pointed out flaws in their reasoning and asserts that the Red List is an invaluable tool to aid conservation efforts. While acknowledging the limitations, \textcite{rodrigues2006value} defended the Red List and denounced some of the previous criticisms as misconceptions. The paper explains that the misconceptions, along with assumptions, prevented the Red List from achieving its full potential. However, the conclusion converges with \textcite{possingham2002limits}, that the Red List should be used in conjunction with other tools to inform conservation. 
% (can expand this paragraph with more details from the paper)

Apart from the discussion surrounding the use of the Red List revolving around human interpretation, two projects have approached the data from a computational perspective. \textcite{gregg2020many} applied sentiment analysis on species common names and was able to conclude that some words were unfavourable as species names \parencite{gregg2020many}, preventing them from attracting sufficient attention and, consequently, resources. Another study used the data and developed a software which incorporated generation length of individual species and status transition rates to simulate future extinctions \parencite{andermann2021iucn_sim}.

With more than 134,400 species assessed, the text information available would require enormous amount of resources to go through in order to extract meaningful information. Advancements in machine learning techniques revolutionised text mining, where conceptual insights can now be obtained by applying these methods to large volumes of text data \parencite{grimmer2013text}. One of the methods is topic analysis, which uses probabilistic models to reveal latent themes and describe patterns in big text data \parencite{blei2012probabilistic}. In 2000, the Red List was digitised \parencite{iucn2021} and the assessments now have to include supporting information for categorising the extinction risk for the species. With the availability of the assessment supporting texts (hereafter 'AST') in plain text, and the inspiration from applying analyses through a computational perspective, an opportunity was created for this project to approach the text data using topic modelling. This study aims to fill the gaps to better inform conservation efforts, where studies like RLI and EDGE fell short of the potential of the Red List by leaving out the big text data. Text-mining by machine reading was applied to discover the latent topics, with a focus on latent as I search for relationships within the collection of AST that are not immediately apparent from reading individual AST. The study will then try to answer the following research questions:
\begin{enumerate}
    \item What are the latent topics in the AST?
    \item How do the topics vary across the different Red List categories?
    \item How do the topics vary across different systems (i.e. terrestrial, marine and freshwater)?
    \item How do the topics appear when mapped to species range data?
\end{enumerate}