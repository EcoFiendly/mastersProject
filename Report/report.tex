\documentclass[a4paper, 11pt]{article}

%\title{Unused Title}
\usepackage{graphicx}
\usepackage{hyperref}
\usepackage{multirow}
\usepackage{multicol}
\usepackage{tabulary}
\usepackage{adjustbox}
\usepackage{blindtext}
\usepackage[utf8]{inputenc}
\usepackage[english]{babel}
\usepackage{csquotes}
\usepackage[T1]{fontenc}

% Use helvet if uarial cannot be installed
%\usepackage{uarial}
\usepackage[scaled]{helvet}

\renewcommand{\familydefault}{\sfdefault}
\usepackage{amssymb}
\usepackage{amsmath}
\usepackage{courier}
\usepackage{setspace}
\usepackage[table,svgnames]{xcolor}
\usepackage{fancyvrb} 
\usepackage{listings}
\usepackage{caption}
\usepackage{longtable}
\usepackage{relsize}
\usepackage{tfrupee}
\usepackage{rotating}
\usepackage{lipsum}
\usepackage{subcaption}
\usepackage{float}
\usepackage{aliascnt}
\usepackage{svg}
\usepackage{placeins}
\usepackage{lineno}
\linenumbers
\usepackage{cleveref}
\captionsetup[table]{labelfont={small,bf,it},textfont={small,it}}
\captionsetup[figure]{labelfont={small,bf,it},textfont={small,it}}

\tolerance=1
\emergencystretch=\maxdimen
\hyphenpenalty=10000
\hbadness=10000

\usepackage[style=authoryear, backend=bibtex]{biblatex}
\bibliography{report.bib}

\makeatletter
\newcommand\footnoteref[1]{\protected@xdef\@thefnmark{\ref{#1}}\@footnotemark}
\makeatother

\newaliascnt{eqfloat}{equation}
\newfloat{eqfloat}{h}{eqflts}
\floatname{eqfloat}{Equation}

\newcommand*{\ORGeqfloat}{}
\let\ORGeqfloat\eqfloat
\def\eqfloat{%
	\let\ORIGINALcaption\caption
	\def\caption{%
		\addtocounter{equation}{-1}%
		\ORIGINALcaption
	}%
	\ORGeqfloat
}

\addto\captionsenglish{% Replace "english" with the language you use
	\renewcommand{\contentsname}%
	{List of Contents}%
}

\newcommand\tab[1][1cm]{\hspace*{#1}}

\definecolor{codegreen}{rgb}{0,0.6,0}
\definecolor{codegray}{rgb}{0.5,0.5,0.5}
\definecolor{codepurple}{rgb}{0.58,0,0.82}
\definecolor{backcolour}{rgb}{0.95,0.95,0.92}

\lstdefinestyle{mystyle}{
	backgroundcolor=\color{backcolour},   
	commentstyle=\color{codegreen},
	keywordstyle=\color{magenta},
	numberstyle=\tiny\color{codegray},
	stringstyle=\color{codepurple},
	basicstyle=\ttfamily\footnotesize,
	breakatwhitespace=false,         
	breaklines=true,                 
	captionpos=b,                    
	keepspaces=true,                 
	numbers=left,                    
	numbersep=5pt,                  
	showspaces=false,                
	showstringspaces=false,
	showtabs=false,                  
	tabsize=2,
	xleftmargin=0.5cm,
	xrightmargin=-0.8cm,
	frame=lr,
	%	framesep=-5pt,
	framerule=0pt
}

\lstset{style=mystyle}

\definecolor{Teal}{RGB}{0,128,128}
\definecolor{NewBlue1}{RGB}{4,100,226}
\definecolor{NiceBlue}{RGB}{63,104,132}
\definecolor{DarkRed}{RGB}{14,53,59}
\definecolor{NewBlue2}{RGB}{62,100,125}
\definecolor{NewBlue3}{RGB}{44,100,128}

\hypersetup{
	colorlinks,
	citecolor=NiceBlue,
	linkcolor=NewBlue1,
	urlcolor=Blue
	%	citebordercolor=Violet,
	%	filebordercolor=Red,
	%	linkbordercolor=Blue
}

\usepackage{geometry}
\linespread{1.25}
\usepackage[parfill]{parskip} % Avoid indentation

\geometry{
	a4paper,
	left=4cm,
	right=2.5cm,
	top=2.5cm,
	bottom=2.5cm,
}


\begin{document}
	\pagenumbering{gobble}
	\begin{center}
		{\large IMPERIAL COLLEGE LONDON}
	\end{center}
	%	\maketitle
	\vspace{6cm}
	
	\begin{center}
		
		\Huge Themes in IUCN Red List assessments revealed by Machine Reading\\		
		\vspace{.5cm}		
		\large {Word Count: 4,735}
		
	\end{center}
	\vspace{2.5cm}
	\begin{center}
		\Large Yew Shen Lim
	\end{center}
	
	\begin{center}
		{\large August 2021}
	\end{center}

	\vspace{7cm}
	\begin{center}
		{\large A thesis submitted in partial fulfilment of the \\requirements for the Master of Research at Imperial College London \\ Sumitted for the MRes in Computational Methods in Ecology and Evolution degree}
	\end{center}		

	\newpage
	\pagenumbering{Roman}

\section*{Declaration}

Both the text data and mammals range data used were downloaded from the IUCN Red List website.

All of the code and graphics were written and created in Python and R. I was responsible for preprocessing the text and conducting the analysis.

My supervisor Dr James Rosindell provided guidance for my project's development in weekly meetings.

\pagebreak

% \section*{Acknowledgments}
% %\include{Acknowledgments.tex}
% \blindtext
% \pagebreak

% \section*{Notation}
% The extent to which you list any symbols used in your report must be left to your discretion. Symbols, which are used in several parts of your report, should preferably be listed before the main text for easy reference. Symbols which are used only once or in one part of the report may be referred to in that part only. Generally, try to place yourself in the position of a reader with average background knowledge and arrange the notation in a manner which will be most convenient for him/her to follow.
% \pagebreak

%\section*{List of Contents}
\tableofcontents
\pagebreak
%\listoffigures
%\listoftables
%\pagebreak
\newpage
%\section{Notation}
%\textbf{IF APPLICABLE}
%%\include{Acknowledgments.tex}


\cleardoublepage\pagenumbering{arabic}


\section*{Abstract}
%\textbf{MAXIMUM 250 WORDS}
The IUCN Red List is the primary source of extinction risk for life on Earth. The availability of the IUCN assessment justifications in plain text format allows for the application of text mining techniques to discover themes and identify potential gaps in existing literature. This study is the first to apply text mining on IUCN assessments. I applied topic modelling on the texts with the aim to discover latent themes. I developed a model using the texts, and the model identified 17 different topics from the texts which were then analysed from different angles. The results of this study complements the literature surrounding biodiversity loss and conservation efforts. This study also seeks to fill the gap between how the Red List is used and how it could be used, laying a foundation for further downstream analyses.
\pagebreak

\section{Introduction}
\subsection{introduction}

The IUCN Red List of Threatened Species\textsuperscript{TM} (hereafter 'Red List') is the primary source of extinction risk for life on Earth.
\pagebreak

\section{Methods}
\subsection{Dataset}

The IUCN Red List AST data was downloaded (December 2020) and the spatial data for mammalian species were downloaded (June 2021) from The IUCN Red List of Threatened Species\textsuperscript{TM} website \parencite{iucn2021}. The AST data contains a large amount of information, and my analysis focuses on a subset of the data which are sentences or paragraphs of text, listed in table \Cref{tab:example and desc of data}.

\begin{table}[hbtp]
    \centering
    \tiny
        \begin{tabulary}{1.0\textwidth}{|L|L|L|L|L|L|L|L|L|}
        \hline
        \multicolumn{2}{|c|}{\textbf{Metadata}} & \multicolumn{7}{c|}{\textbf{Text data to be preprocessed}}\\
        \hline
        Red List Category & Systems & Rationale & Habitat & Threats & Population & Range & Use and trade & Conservation actions\\
        \hline
        Lists the assigned category for the species & Terrestrial, marine and freshwater (or combinations of the three) & Summary of the text justification for the category assigned & General habitat use and ecology & General and specific threats that are affecting the species & Population numbers, structure and current population trend & Historical and current geographic range & How species are used and/or traded & Current and recommended general and specific actions in place to protect the species\\
        \hline
        \end{tabulary}
    \small
    \caption{Example and description of the data used in my analysis (see \parencite{iucn2021}). The first two columns are metadata that will be later used to group the results from the model. The next seven columns are text data that will be preprocessed and used to train the model to reveal the "topics". Each row of information represents the data collected by the IUCN for a single species. Each of the entries in the last seven columns is equivalent to one document (LDA terminology)}
    \label{tab:example and desc of data}
\end{table}

\subsection{Data preprocessing}

The texts were processed in multiple steps. Only global assessments in English were selected. This removes noise from regional assessments and assessments not recorded in English. Texts were then cleaned using regular expressions. The texts were then processed using spaCy \parencite{spacy}, where stop words were removed, remaining words were lemmatised and tokenised, producing a bag of words (BoW) matrix. The phraser module from Gensim \parencite{rehurek_lrec} was then applied onto the BoW to detect and generate bigrams (pairs words that when occurring together could have a different meaning, e.g., global and warming forming global warming). Finally, a dictionary (vocabulary) of the tokens (words or bigrams) was created. Tokens that appeared in less than 1\% or more than 50\% of documents were filtered out for being too rare or common and thus uninformative. The threshold of tokens appearing in more 50\% was selected as this study is looking for latent themes.

\subsection{Model, terminology and selection}

Latent Dirichlet Allocation (LDA) \parencite{blei2003latent} was used to perform the topic analysis. LDA is a type of generative probabilistic model under the class of unsupervised machine learning. LDA is able to reduce the dimensionality of large amounts of text data, and identify \textit{"topics"} from within based on words that are statistically associated. The algorithm estimates, for a given number \textit{k} of topics, the probability of each word belonging to each topic, and the probability of a word from a document belonging to a topic. This means that the algorithm assumes that each document is generated by a collection of topics and topics contain keywords that occur together often. We can then infer what the document is about from the topics that are discovered by the model.

The basic unit is a word or bigram (or token) which is present in the constructed dictionary (preprocessing step); a document is made up of any number of words; a corpus is a collection of any number of documents \parencite{blei2003latent}. In relation to my dataset, the corpus is the collection of all the text data from the last seven columns in \Cref{tab:example and desc of data}, and each entry from the seven sections is a document, and is made up of a number of words. As such, each species would comprise of seven documents, and the length of the corpus is number of species multiplied by seven.

Models were trained with the number of topics ranging from 2 to 100 using the prepared corpus. Their performance was measured using topic coherence as a metric \parencite{roder2015exploring}. On top of the computed metrics (\Cref{fig:cv plot}), I inspected the topic space that was visualised using pyLDAvis (a python port of LDAvis \parencite{sievert2014ldavis}). The model selection suggested 17 topics as the optimal number for this corpus. The labels and top 16 words for each topic are shown in \Cref{tab:topic label and top 16 words}.

Using the selected model, topic compositions for each of the documents were calculated. Topics can have a non-zero probability of being discovered for every document, thus not all of the topics in each document will neccessarily be informative, especially when the probibility is very small. In order to focus on the topics that are informative, the topic probabilities were plotted against their probability density (\Cref{fig:kde plot}). This revealed that a threshold of 1/17 was suitable for deciding if a document was about that topic, where probabilities below the threshold were masked to zero.

I calculated the intertopic distance using four different distance measures Kullback-Leibler divergence, Hellinger distance, Jaccard Index and Jensen-Shannon divergence. Ward's linkage \parencite{ward1963hierarchical}, a hierarchical clustering technique, was then applied on the four different measure outputs to minimise the topic distances and cluster the topics accordingly (\Cref{fig:dendrograms}). I selected the cluster generated from the Jaccard's index (\Cref{fig:topic heatmap}) after visually comparing the different clusters by plotting them onto dendrograms. 

Due to the nature of LDA, words can be and were assigned to multiple topics, e.g. the word threat was found in six different topics. This is a limitation of LDA, and results from the model selection. If there are too few topics, the topics themselves might not provide a clear meaning. However, if there are too many topics, it could result in many junk topics and only a smaller numbr of topics that will be useful.

\subsection{Aggregating the data by IUCN metadata}
The document probabilities were aggregated for every species. These were then grouped in four different ways to produce heatmaps of the topic probabilities to each of the groups. The first three groups are the seven sections, Red List category and Systems, which are IUCN metadata, and the fourth is taxonomic class from the IUCN.

\subsection{Mapping topics to mammalian species range}

The analysis is further expanded by mapping topics to mammalian species range. I selected the mammalian species range to start with because of the focus of literature on mammals \parencite{cardillo2008predictability,gonzalez2013intrinsic,purvis2000predicting} compared to other species such as plants \parencite{stefanaki2015lessons,powney2014phylogenetically,sodhi2008correlates} and invertebrates \parencite{arbetman2017global,koh2004ecological,sullivan2000comparative}. These are mammalian range data obtained from the IUCN, and does not contain every mammalian species. Using the previous threshold of 1/17, the topic probabilities were converted to presence and absence values represented by 1 and 0. These values were then summed and averaged to determine the ratio of mammalian species with the topic present to the total number of mammalian species present at each pixel. For example, if there are 15 species present on a particular pixel, and only three of them have a topic present, the ratio would be 1/5 at that pixel.

\pagebreak

\section{Results}
\subsection{Latent Topics}

(Table of topics and their corresponding words)

Describe the model output

\subsection{Topics by Red List Category}

(Figure of heatmap for red list category)

\subsection{Topics by systems}

(Figure of heatmap for systems)

\subsection{Topics by realms}

(Figure of heatmap for realms)

\pagebreak

\section{Discussion}
% partition into subsections:

In this paper, I explored the application of rapidly summarising topics on the IUCN AST. My results suggest that this was successful in discovering themes from the texts and help better understand relationships within the AST as a whole, without the need to go through every individual assessment. This method could be used alongside meta-analyses on the IUCN's data in the future to generate additional insights.

% (explaining and writing down all of the thought process I go through)

\subsection{Latent topics in the supporting texts}

The latent topics in the supporting texts revealed by the model demonstrates some fundamental use of language that can even be detected by machines. The intention of the species AST summaries is apparent when considering the topics featured in the Rationale section of the Red List assessments (\Cref{fig:topic heatmap}). Red List categories are given based largely on the population and range \parencite{iucn2012}, this relates to how the information are selected and shown on the summaries. The summaries let readers promptly find out the rationale behind the species' given Red List category. The description of the Habitat section (\Cref{tab:example and desc of data}) corresponds directly to the names of the two ecology topics "Terrestrial ecology" and "Aquatic ecology", which are also the top two topics in the section. Habitat loss is the main driver of biodiversity loss \parencite{diaz2019pervasive, pimm2014biodiversity}, thus would be mentioned frequently in the Habitat section for species AST, which was detected by the model and revealed through the top three topics for the Threats section (\Cref{fig:topic heatmap}). As for the Population section, the two of the top three topics (\Cref{fig:topic heatmap}) describe population, and the top topic is "Common threats". The description of the Population section (\Cref{tab:example and desc of data}) shows that the population trend is described, which could be a reason why "Common threats" is the top topic as threats are intertwined with population trends. However, "Common threats" could also be mislabeled as the tokens within the topic (\Cref{tab:topic label and top 16 words}) lean toward population rather than threats, despite the second top token being "threat". The top two topics in Range section also properly describe the section, being the only two topics that are about range. The top two topics in the Use and trade section brings an interesting point to attention. When thinking about the two topics "Habitat loss" and "Fisheries threats" and the phrase "Use and trade", one would think of fisheries threats being closer related to the phrase compared to habitat loss. However, the use and trade of plant species are logically contributing to habitat loss as the plants themselves could play the role of a habitat for other species. Finally, in the Conservation actions section, the top two topics are conservation related and have been detected by the model.

\subsection{Latent topics in the Red List categories, systems and taxonomic classes}

Across Red List categories, the probability of the "Spatial data" topic increases down the categories. This shows that spatial data of species are increasingly mentioned in the AST as the category assigned increases in severity, all the way until the species is categorised as extinct. This relates back to how the species are categorised, where the spatial data of the species play a large role \parencite{iucn2012}. The probability of "Habitat loss" being almost uniform across all categories also relates back to how habitat loss the main driver of biodiversity loss \parencite{diaz2019pervasive, pimm2014biodiversity}.

AST of species in the \textit{Anthozoa} (includes sea anemones, and corals) class prominently featuring "Marine population structure" is presumably related to the topic containing the tokens "coral" and "reef" (\Cref{tab:topic label and top 16 words}). Beyond that, the discussion of population structure of Anthozoans is doubtlessly due to the destruction of coral reefs all over the world \parencite{Johnson2020CoralRD,hoegh2007coral,hughes2007phase,hughes2003climate}.

The topic "Fisheries threats" standing out for \textit{Chondrichthyes}, cartilaginous fishes, is consistent with their population decline \parencite{pacoureau2021half}, and studies into use and trade for them \parencite{stevens2000effects}. Sharks are not only exploited for their fins, squalene or squalane is an oil made from Shark's liver and used in a wide range of cosmetic products, lubricants and cleaning agents \parencite{naziri2011squalene,vannuccini1999shark,gopakumar1986squalene}. Shark cartilage is a popular dietary supplement despite lacking in scientific evidence \parencite{utsunomiya2003shark}. Shark and ray meat are marketed under many different names, Flake, Huss (served in fish and chips stores in the UK), Rock Salmon, Sea Ham, and many more \parencite{vannuccini1999shark}. Contrary to popular belief, the market for sharks trade is not just limited to sharks fin in Asia, the European Union is responsible for more than twenty percent of global trade in shark meat \parencite{wwf2021shark}.

\subsection{Latent topics on mammalian species range}
By mapping the topic presence and absence onto mammalian species range data, an overview of where the topics are geographically prominent can be observed (\Crefrange{fig:hab los theme}{fig:eco con act theme}).

One would expect the topic "Forest loss" to appear much stronger in the Amazon rainforest, as there is constant deforestation happening in the Amazon \parencite{Ellwanger2020BeyondDL,Escobar2020defor}. However, this is not reflected in on the map in my results (\Cref{subfig:for los}). First, there are range data of about 160 mammals that range over the Amazon rainforest, and about 100 of them have their AST contain the "Forest loss" topic which results in a proportion of about 0.6. Second, there are about 45500 species living in the Amazon \parencite{da2005fate}, which means that my result is only representative of 0.003\% of the species that live in the Amazon. This shows that "Habitat loss" could reflect much stronger in the Amazon if range data of other species were incorporated. 

The "Water pollution threat" topic is prominent globally and reflects very strongly in developed parts of the world (North America, Europe, Australia and New Zealand) (\Cref{subfig:wat pol}). Additionally, the topic is also quite high in proportion, at more than 0.75, in developing regions (Africa and China). This shows a strong relationship between this topic and freshwater (inland waters), and anthropogenic activity. Development comes at a cost, and anthropogenic activity contributes to pollution, not just of water \parencite{zhou2020e00925,emmanuel2016new} but the global environment as a whole \parencite{Brusseau2019TheEO}. Although water pollution is also heavy in the seas \parencite{Barrett2020MicroplasticPI,Zhang2020MicroplasticPI,Eriksen2014PlasticPI}, the relatively weaker reflection from my results could again be drawn back to the lack of non mammalian species range data incorporated in the analysis.

The "Habitat loss" topic reflecting strongly on all terrestrial regions puts additional emphasis on habitat loss being the main driver of biodiversity loss \parencite{diaz2019pervasive, pimm2014biodiversity}. The topic "Forest fragmentation" reflecting strongly in South America shows that the use of language could be affecting how the topics are detected in the AST. This relates to the lower presence of "Forest loss" in the Amazon, as previously discussed, which could be a subtle difference in the language use and reveals that forest fragmentation is the main issue rather than forest loss in the Amazon. One might think that fragmentation and loss describe the same issue but they are in fact different, however fragmentation can be a subset of loss \parencite{liu2016socioeconomic,fahrig2003effects}.

The high proportion of topics "Marine population structure" and "Spatial data" globally (\Cref{subfig:mar pop} and \Cref{subfig:spa dat}) shows that these two topics underpin many of the mammalian AST, and again shows how population structure and spatial data are main factors contributing to a species' Red List category. This points to the language used in AST of marine mammals as they were detected by the model and grouped together with words that are used often in the AST of marine species. The lower proportion for topics "Population dynamics" and "Range, occurrence and occupancy" (\Cref{subfig:pop dyn} and \Cref{subfig:ran occ occ}) compared to the previous two, show that the model has also picked up the order of Red List criteria, where the amongst the criteria, population size comes before range \parencite{iucn2012}.

The topics "Terrestrial ecology" and "Aquatic ecology" (\Cref{subfig:ter eco} and \Cref{subfig:aqu eco}) also reflects the performance of the model, where it was able to distinguish the choice of words used between the AST of terrestrial and aquatic mammals. The topics "Conservation actions" and "Monitoring and rewilding" standing out globally (\Cref{subfig:con act} and \Cref{subfig:mon rew}) shows that almost all of the AST of these mammals mention these topics. This reminds us of how much biodiversity is under threat \parencite{butchart2010global}, where almost all of these mammals require conservation actions. The "Monitoring and rewilding" topic also reminds us how much the range of these mammals have been affected and that one of the main actions proposed in the AST is to continue to monitor their numbers and rewild whenever possible \parencite{thakur2020climate,perino2019rewilding,corlett2016restoration}. The topic "Area based protection" showing up mostly on terrestrial regions (\Cref{subfig:are bas pro}) shows that area based protection is mentioned mostly in terrestrial mammals. Area protection is a classification of conservation measure by the IUCN Red List \parencite{dudley2008guidelines} which involves setting up or expanding areas such as national parks, nature reserves, marine protected areas and others that would contribute to species conservation. The topic not showing up in the oceans could also be related to the small number of oceanic marine mammals. Additionally, there are no barriers restricting the movement of marine mammals and they are often migratory which makes it difficult in setting up protected areas for them \parencite{Lascelles2014MigratoryMS,Freeman2003DISTRIBUTIONOH}. There is also the difficulty of setting up area protection in the oceans, where protecting areas in the oceans would require vast amounts of resources. Monitoring the areas that are far away from land is difficult. Defense and deterence for the areas would also be required, and setting up strategies for the areas would be counterproductive as that would contribute to disturbance of the local biodiversity.

\subsection{Limitations}
Marine mammals are not representative of oceanic species, if other range data were available that would better represent topic presence in the oceanic regions. However, the concepts presented can be extrapolated to hypothesise the patterns for others species, where topic prominence plotted spatially would not be limited to mammals within the respective space. It would be likely that topics such as "Spatial data", "Conservation actions" and "Water pollution threat" reflect just as or even stronger if the other species were included in the analysis.

Despite my goal to discover latent topics, the topics that were discovered by the models do not seem to be latent. This could result from the thresholds that I had set for the dictionary, where tokens appearing in more than 50\% of documents were removed being too high. Future work can try a lower threshold if they intend to look into latent themes. However, it could also result from the language use in the AST, where it follows a strict set of guidelines thus making latent themes much harder to discover.
\pagebreak

\section{Conclusion}
The findings of this study can be the foundation for prospective research. First, the inclusion of range data from other species could improve the resolution of the maps of topic presence in proportion to species present. The amount of information revealed from mapping topics to the range data of only 11842 mammalian species is already informative, hopefully a bigger picture can be put together by expanding the maps to include the range data of other species available from the IUCN. Second, a separate model could be trained with the intention of comparing the resulting topic distribution of the Rationale section with the topic distribution of the full AST to measure how well the summaries capture each of the AST. Additionally, as I proposed in the introduction, these results can be used in conjunction with meta analyses conducted with the IUCN's data to gain more meaningful insights, to better influence conservation efforts. I hope that my study sets a precedence for further application of natural language processing methods in the Red List AST. Along with how fast the field of natural language processing is advancing, I believe there are more to uncover in this large trove of plain text information, either through the use of state of the art models and/or through another pair of curious and creative lenses. 
\pagebreak

\section{References}
\printbibliography[heading=none]
\pagebreak

\pagenumbering{roman}

\section{Appendix}
\renewcommand\thefigure{\thesection.\arabic{figure}}  
\setcounter{figure}{0}
\begin{figure}[!htbp]
    \centering
    \includegraphics[width=\linewidth]{../Data/comb_global/coherencePlot.png}
    \small
    \caption{Plot of the C\_v metric score against the number of topics selected, ranging from 2 to 100.}
    \label{fig:cv plot}
\end{figure}

\begin{figure}[!htbp]
    \centering
    \includesvg[width=\linewidth]{../Data/comb_global/comb_kde.svg}
    \small
    \caption{KDE plot of topic probabilities against their probability density}
    \label{fig:kde plot}
\end{figure}

\begin{figure}[!htbp]
    \centering
    \includesvg[width=\linewidth]{../Data/comb_global/four_dendrogram.svg}
    \small
    \caption{Dendrograms by applying Ward's linkage to different distance measures}
    \label{fig:dendrograms}
\end{figure}

\end{document}   