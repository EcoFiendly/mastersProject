\documentclass[a4paper, 11pt]{article}

%\title{Unused Title}
\usepackage{graphicx}
\usepackage{hyperref}
\usepackage{multirow}
\usepackage{multicol}
\usepackage{tabulary}
\usepackage{adjustbox}
\usepackage{blindtext}
\usepackage[utf8]{inputenc}
\usepackage[english]{babel}
\usepackage{csquotes}
\usepackage[T1]{fontenc}

% Use helvet if uarial cannot be installed
%\usepackage{uarial}
\usepackage[scaled]{helvet}

\renewcommand{\familydefault}{\sfdefault}
\usepackage{amssymb}
\usepackage{amsmath}
\usepackage{courier}
\usepackage{setspace}
\usepackage[table,svgnames]{xcolor}
\usepackage{fancyvrb} 
\usepackage{listings}
\usepackage{caption}
\usepackage{longtable}
\usepackage{relsize}
\usepackage{tfrupee}
\usepackage{rotating}
\usepackage{lipsum}
\usepackage{subcaption}
\usepackage{float}
\usepackage{aliascnt}
\usepackage{svg}
\usepackage{placeins}
\usepackage{lineno}
\linenumbers
\usepackage{cleveref}
\captionsetup[table]{labelfont={small,bf,it},textfont={small,it}}
\captionsetup[figure]{labelfont={small,bf,it},textfont={small,it}}

\tolerance=1
\emergencystretch=\maxdimen
\hyphenpenalty=10000
\hbadness=10000

\usepackage[style=authoryear, backend=bibtex]{biblatex}
\bibliography{report.bib}

\makeatletter
\newcommand\footnoteref[1]{\protected@xdef\@thefnmark{\ref{#1}}\@footnotemark}
\makeatother

\newaliascnt{eqfloat}{equation}
\newfloat{eqfloat}{h}{eqflts}
\floatname{eqfloat}{Equation}

\newcommand*{\ORGeqfloat}{}
\let\ORGeqfloat\eqfloat
\def\eqfloat{%
	\let\ORIGINALcaption\caption
	\def\caption{%
		\addtocounter{equation}{-1}%
		\ORIGINALcaption
	}%
	\ORGeqfloat
}

\addto\captionsenglish{% Replace "english" with the language you use
	\renewcommand{\contentsname}%
	{List of Contents}%
}

\newcommand\tab[1][1cm]{\hspace*{#1}}

\definecolor{codegreen}{rgb}{0,0.6,0}
\definecolor{codegray}{rgb}{0.5,0.5,0.5}
\definecolor{codepurple}{rgb}{0.58,0,0.82}
\definecolor{backcolour}{rgb}{0.95,0.95,0.92}

\lstdefinestyle{mystyle}{
	backgroundcolor=\color{backcolour},   
	commentstyle=\color{codegreen},
	keywordstyle=\color{magenta},
	numberstyle=\tiny\color{codegray},
	stringstyle=\color{codepurple},
	basicstyle=\ttfamily\footnotesize,
	breakatwhitespace=false,         
	breaklines=true,                 
	captionpos=b,                    
	keepspaces=true,                 
	numbers=left,                    
	numbersep=5pt,                  
	showspaces=false,                
	showstringspaces=false,
	showtabs=false,                  
	tabsize=2,
	xleftmargin=0.5cm,
	xrightmargin=-0.8cm,
	frame=lr,
	%	framesep=-5pt,
	framerule=0pt
}

\lstset{style=mystyle}

\definecolor{Teal}{RGB}{0,128,128}
\definecolor{NewBlue1}{RGB}{4,100,226}
\definecolor{NiceBlue}{RGB}{63,104,132}
\definecolor{DarkRed}{RGB}{14,53,59}
\definecolor{NewBlue2}{RGB}{62,100,125}
\definecolor{NewBlue3}{RGB}{44,100,128}

\hypersetup{
	colorlinks,
	citecolor=NiceBlue,
	linkcolor=NewBlue1,
	urlcolor=Blue
	%	citebordercolor=Violet,
	%	filebordercolor=Red,
	%	linkbordercolor=Blue
}

\usepackage{geometry}
\linespread{1.25}
\usepackage[parfill]{parskip} % Avoid indentation

\geometry{
	a4paper,
	left=4cm,
	right=2.5cm,
	top=2.5cm,
	bottom=2.5cm,
}


\begin{document}
	\pagenumbering{gobble}
	\begin{center}
		{\large IMPERIAL COLLEGE LONDON}
	\end{center}
	%	\maketitle
	\vspace{6cm}
	
	\begin{center}
		
		\Huge Themes in IUCN Red List assessments revealed by Machine Reading\\		
		\vspace{.5cm}		
		\large {Word Count: 4,735}
		
	\end{center}
	\vspace{2.5cm}
	\begin{center}
		\Large Yew Shen Lim
	\end{center}
	
	\begin{center}
		{\large August 2021}
	\end{center}

	\vspace{7cm}
	\begin{center}
		{\large A thesis submitted in partial fulfilment of the \\requirements for the Master of Research at Imperial College London \\ Sumitted for the MRes in Computational Methods in Ecology and Evolution degree}
	\end{center}		

	\newpage
	\pagenumbering{Roman}

\section*{Declaration}

Both the text data and mammals range data used were downloaded from the IUCN Red List website.

All of the code and graphics were written and created in Python and R. I was responsible for preprocessing the text and conducting the analysis.

My supervisor Dr James Rosindell provided guidance for my project's development in weekly meetings.

\pagebreak

% \section*{Acknowledgments}
% %\include{Acknowledgments.tex}
% \blindtext
% \pagebreak

% \section*{Notation}
% The extent to which you list any symbols used in your report must be left to your discretion. Symbols, which are used in several parts of your report, should preferably be listed before the main text for easy reference. Symbols which are used only once or in one part of the report may be referred to in that part only. Generally, try to place yourself in the position of a reader with average background knowledge and arrange the notation in a manner which will be most convenient for him/her to follow.
% \pagebreak

%\section*{List of Contents}
\tableofcontents
\pagebreak
%\listoffigures
%\listoftables
%\pagebreak
\newpage
%\section{Notation}
%\textbf{IF APPLICABLE}
%%\include{Acknowledgments.tex}


\cleardoublepage\pagenumbering{arabic}


\section*{Abstract}
%\textbf{MAXIMUM 250 WORDS}
The IUCN Red List is the primary source of extinction risk for life on Earth. The availability of the IUCN assessment justifications in plain text format allows for the application of text mining techniques to discover themes and identify potential gaps in existing literature. This study is the first to apply text mining on IUCN assessments. I applied topic modelling on the texts with the aim to discover latent themes. I developed a model using the texts, and the model identified 17 different topics from the texts which were then analysed from different angles. The results of this study complements the literature surrounding biodiversity loss and conservation efforts. This study also seeks to fill the gap between how the Red List is used and how it could be used, laying a foundation for further downstream analyses.
\pagebreak

\section{Introduction}
The International Union for Conservation of Nature's (IUCN) Red List of Threatened Species\textsuperscript{TM} (hereafter 'Red List') is the primary source of extinction risk for life on Earth \parencite{iucn2021, rodrigues2006value, lamoreux2003value}. Although officially established in 1964 by The IUCN, the Red List dates back to a year prior in 1963 \parencite{scott1987red}. It started off as the International Red Data Books, which recorded threatened wildlife along with the threats they faced. Now, the IUCN states that the Red List is "Far more than a list of species and their status" \parencite{iucn2021}. 

Between now and its inception, the Red List had faced a number of criticisms. \textcite{master1991assessing} criticised the old extinction risk categories for lack of clarity and direction and discusses supplementary ranking methodologies. \textcite{mrosovsky1997iucn} questioned the Red List's credibility due to the witholding of key information behind decision making for some species. \textcite{possingham2002limits} emphasises that these lists should be used as a supplement rather than key factor in informing resource allocation (optimised by targetting species which can recover more using the same amount of resources, regardless of extinction category); reserve selection (look at all species occupying the selected area rather than select areas based on umbrella species); development or exploitation (listed/unlisted species can affect how resources are exploited [difficulty with explaination, ask for help]); reflect change in status of biodiversity (skewed representation of taxa results in accurate reflection of situation). \textcite{lamoreux2003value} largely agreed with \textcite*{possingham2002limits} but pointed out flaws in their reasoning and asserts that the Red List is an invaluable tool to aid conservation efforts. 

While acknowledging the limitations, \textcite{rodrigues2006value} defended the Red List and denounces some of the previous criticisms as misconceptions. The paper explains that the misconceptions, along with assumptions, prevented the Red List from achieving its potential. However, the conclusion converges with \textcite{possingham2002limits}, that the Red List should be used in conjunction with other tools to inform conservation. (can expand this paragraph with more details from the paper)

With the discussion surrounding the use of the Red List revolving around human interpretation, two projects have approached the data from a computational perspective. \textcite{gregg2020many} applied sentiment analysis on species common names and was able to conclude that some words were unfavourable as species names \parencite{gregg2020many}, preventing them from attracting sufficient attention and consequently, resources. Another research used the data and developed a software which incorporated generation length of individual species and status transition rates to simulate future extinctions \parencite{andermann2021iucn_sim}.

In 2000, the Red List was digitised \parencite{iucn2021} and the assessments now have to include supporting information for categorising the extinction risk for the species.
The availability of the information in plain text creates the opportunity for this project to approach the data from a computational perspective. Text-mining by machine reading was applied to try to answer the following research questions:
\begin{enumerate}
    \item What are the latent topics in the supporting texts.
    \item How do the topics vary across the different Red List categories.
    \item How do the topics vary across different systems (i.e. terrestrial, marine and freshwater)
    \item How do the topics vary across different biogeographic realms.
\end{enumerate}

\pagebreak

\section{Methods}
\subsection{Dataset}

The IUCN Red List AST data was downloaded (December 2020) and the spatial data for mammalian species were downloaded (June 2021) from The IUCN Red List of Threatened Species\textsuperscript{TM} website \parencite{iucn2021}. The AST data contains a large amount of information, and my analysis focuses on a subset of the data which are sentences or paragraphs of text, listed in table \Cref{tab:example and desc of data}.

\begin{table}[hbtp]
    \centering
    \tiny
        \begin{tabulary}{1.0\textwidth}{|L|L|L|L|L|L|L|L|L|}
        \hline
        \multicolumn{2}{|c|}{\textbf{Metadata}} & \multicolumn{7}{c|}{\textbf{Text data to be preprocessed}}\\
        \hline
        Red List Category & Systems & Rationale & Habitat & Threats & Population & Range & Use and trade & Conservation actions\\
        \hline
        Lists the assigned category for the species & Terrestrial, marine and freshwater (or combinations of the three) & Summary of the text justification for the category assigned & General habitat use and ecology & General and specific threats that are affecting the species & Population numbers, structure and current population trend & Historical and current geographic range & How species are used and/or traded & Current and recommended general and specific actions in place to protect the species\\
        \hline
        \end{tabulary}
    \small
    \caption{Example and description of the data used in my analysis (see \parencite{iucn2021}). The first two columns are metadata that will be later used to group the results from the model. The next seven columns are text data that will be preprocessed and used to train the model to reveal the "topics". Each row of information represents the data collected by the IUCN for a single species. Each of the entries in the last seven columns is equivalent to one document (LDA terminology)}
    \label{tab:example and desc of data}
\end{table}

\subsection{Data preprocessing}

The texts were processed in multiple steps. Only global assessments in English were selected. This removes noise from regional assessments and assessments not recorded in English. Texts were then cleaned using regular expressions. The texts were then processed using spaCy \parencite{spacy}, where stop words were removed, remaining words were lemmatised and tokenised, producing a bag of words (BoW) matrix. The phraser module from Gensim \parencite{rehurek_lrec} was then applied onto the BoW to detect and generate bigrams (pairs words that when occurring together could have a different meaning, e.g., global and warming forming global warming). Finally, a dictionary (vocabulary) of the tokens (words or bigrams) was created. Tokens that appeared in less than 1\% or more than 50\% of documents were filtered out for being too rare or common and thus uninformative. The threshold of tokens appearing in more 50\% was selected as this study is looking for latent themes.

\subsection{Model, terminology and selection}

Latent Dirichlet Allocation (LDA) \parencite{blei2003latent} was used to perform the topic analysis. LDA is a type of generative probabilistic model under the class of unsupervised machine learning. LDA is able to reduce the dimensionality of large amounts of text data, and identify \textit{"topics"} from within based on words that are statistically associated. The algorithm estimates, for a given number \textit{k} of topics, the probability of each word belonging to each topic, and the probability of a word from a document belonging to a topic. This means that the algorithm assumes that each document is generated by a collection of topics and topics contain keywords that occur together often. We can then infer what the document is about from the topics that are discovered by the model.

The basic unit is a word or bigram (or token) which is present in the constructed dictionary (preprocessing step); a document is made up of any number of words; a corpus is a collection of any number of documents \parencite{blei2003latent}. In relation to my dataset, the corpus is the collection of all the text data from the last seven columns in \Cref{tab:example and desc of data}, and each entry from the seven sections is a document, and is made up of a number of words. As such, each species would comprise of seven documents, and the length of the corpus is number of species multiplied by seven.

Models were trained with the number of topics ranging from 2 to 100 using the prepared corpus. Their performance was measured using topic coherence as a metric \parencite{roder2015exploring}. On top of the computed metrics (\Cref{fig:cv plot}), I inspected the topic space that was visualised using pyLDAvis (a python port of LDAvis \parencite{sievert2014ldavis}). The model selection suggested 17 topics as the optimal number for this corpus. The labels and top 16 words for each topic are shown in \Cref{tab:topic label and top 16 words}.

Using the selected model, topic compositions for each of the documents were calculated. Topics can have a non-zero probability of being discovered for every document, thus not all of the topics in each document will neccessarily be informative, especially when the probibility is very small. In order to focus on the topics that are informative, the topic probabilities were plotted against their probability density (\Cref{fig:kde plot}). This revealed that a threshold of 1/17 was suitable for deciding if a document was about that topic, where probabilities below the threshold were masked to zero.

I calculated the intertopic distance using four different distance measures Kullback-Leibler divergence, Hellinger distance, Jaccard Index and Jensen-Shannon divergence. Ward's linkage \parencite{ward1963hierarchical}, a hierarchical clustering technique, was then applied on the four different measure outputs to minimise the topic distances and cluster the topics accordingly (\Cref{fig:dendrograms}). I selected the cluster generated from the Jaccard's index (\Cref{fig:topic heatmap}) after visually comparing the different clusters by plotting them onto dendrograms. 

Due to the nature of LDA, words can be and were assigned to multiple topics, e.g. the word threat was found in six different topics. This is a limitation of LDA, and results from the model selection. If there are too few topics, the topics themselves might not provide a clear meaning. However, if there are too many topics, it could result in many junk topics and only a smaller numbr of topics that will be useful.

\subsection{Aggregating the data by IUCN metadata}
The document probabilities were aggregated for every species. These were then grouped in four different ways to produce heatmaps of the topic probabilities to each of the groups. The first three groups are the seven sections, Red List category and Systems, which are IUCN metadata, and the fourth is taxonomic class from the IUCN.

\subsection{Mapping topics to mammalian species range}

The analysis is further expanded by mapping topics to mammalian species range. I selected the mammalian species range to start with because of the focus of literature on mammals \parencite{cardillo2008predictability,gonzalez2013intrinsic,purvis2000predicting} compared to other species such as plants \parencite{stefanaki2015lessons,powney2014phylogenetically,sodhi2008correlates} and invertebrates \parencite{arbetman2017global,koh2004ecological,sullivan2000comparative}. These are mammalian range data obtained from the IUCN, and does not contain every mammalian species. Using the previous threshold of 1/17, the topic probabilities were converted to presence and absence values represented by 1 and 0. These values were then summed and averaged to determine the ratio of mammalian species with the topic present to the total number of mammalian species present at each pixel. For example, if there are 15 species present on a particular pixel, and only three of them have a topic present, the ratio would be 1/5 at that pixel.

\pagebreak

\section{Results}
\subsection{Topic analysis}
% talk about model selection, generality vs specificity

% (write it presenting the key idea table 2 talks about)

Model selection converged at 17 topics from analysing the corpus (\Cref{tab:topic label and top 16 words}). The topics were then clustered into five themes: Threats, Population, Ecology, Range, Conservation Actions. The Threats theme contains six topics. These six topics are futher divided into two subgroups, one that is describing loss of habitat ("Forest loss", "Water pollution threat" and "Habitat loss") and the other describing the classification of threats ("Threat distribution" and "Common threats"). The Population theme contains three topics ("Population dynamics", "Marine population structure" and "Assessment criteria") that describe population status, and population status is a direct and major factor that affects how species are assessed for their Red List category \parencite{iucn2012}. The Ecology theme contains topics ("Terrestrial ecology", "Aquatic ecology" and "Fisheries threats") that are about ecology. The Range theme contains two topics ("Range, occurrence and occupancy" and "Spatial data") that relate to spatial distribution, where occurrence outlines a perimeter that samples can be observed and occupancy is how much of the space within the perimeter is occupied. The Conservation Actions theme contains topics ("Conservation Actions", "Area based protection" and "Monitoring and rewilding") that describe conservation actions.

\begin{table}[!hbtp]
\centering
    \small
    \begin{tabulary}{1.0\textwidth}{L L}
        \hline
        Topic & Top 16 Words\\
        \hline
        Forest loss & forest, area, habitat, agriculture, threat, fire, threaten, plant, mining, plantation, timber, deforestation, tree, logging, wood, activity\\
        \hline
        Water pollution threat & threat, habitat, impact, human, water, pollution, activity, development, increase, population, major, cause, change, potential, use, area\\
        \hline
        Habitat loss & habitat, loss, species, threat, degradation, range, threaten, area, forest, population, likely, deforestation, assess, decline, occur, agriculture\\
        \hline
        Forest fragmentation & forest, fragment, habitat, threaten, severely, area, taxon, population, decline, rate, subpopulation, range, consider, category, distribution, land\\
        \hline
        Threat distribution & threat, distribution, wide, large, major, population, currently, assess, significant, future, identify, tree, view, concern, presume, range\\
        \hline
        Common threats & common, threat, population, know, widespread, major, abundant, locally, range, information\_available, list, stable, widely\_distribute, fairly, relatively, consider\\
        \hline
        Population dynamics & population, size, trend, range, criterion, approach\_threshold, decline, year\_generation, believe, quantify, large, small, estimate, number, mature\_individual, appear\\
        \hline
        Assessment criteria & decline, population, habitat, criterion, quality, assess, range, extent, number, ongoing, location, suspect, small, extent\_occurrence, qualify, occur\\
        \hline
        Marine population structure & population, individual, estimate, decline, year, number, coral, subpopulation, reef, total, generation, mature\_individual, range, increase, reduction, base\\
        \hline
        Terrestrial ecology & forest, find, tree, occur, grow, species, area, lowland, dry, habitat, small, montane, soil, shrub, altitude, open\\
        \hline
        Aquatic ecology & water, find, depth, river, occur, stream, small, female, male, species, length, inhabit, area, large, egg, shallow\\
        \hline
        Fisheries threats & trade, fishery, use, catch, information, fishing, datum, fish, commercial, know, report, available, harvest, population, range, target\\
        \hline
        Range, occurrence and occupancy & know, area, locality, location, extent\_occurrence, occupancy, km², estimate, endemic, collection, threat, type, base, habitat, occur, find\\
        \hline
        Spatial data & record, endemic, occur, know, island, find, range, species, asl, area, province, elevation, specimen, region, report, recent\\
        \hline
        Conservation actions & conservation, need, population, research, action, protect, area, management, habitat, trend, distribution, threat, protection, ecology, recommend, require\\
        \hline
        Area based protection & protect, area, occur, conservation\_measure, know, specific, place, species, collection, range, present, list, utilize, find, subpopulation, marine\\
        \hline
        Monitoring and rewilding & population, area, survey, habitat, conservation, range, status, suitable, protect, local, study, site, country, current, establish, need\\
        \hline
        \end{tabulary}
    \caption{Topics and their top 16 words revealed by the LDA model. Topic labels were assigned after visual inspection of the words in each of the topics and visualisation of their relationship according to how they were clustered using Ward's linkage and visualised using IUCN metadata \Cref{fig:topic heatmap}}
    \label{tab:topic label and top 16 words}
\end{table}
\FloatBarrier

\subsection{Topic composition by IUCN metadata and taxonomic class}

\begin{figure}[!htbp]
\centering
\includesvg[width=\linewidth]{../Data/comb_global/dend_comb_hm.svg}
\small
\caption{Heatmaps of logged probabilities when the documents were grouped by three different IUCN metadata and taxonomic class. The dendrogram on top shows the intertopic relationships. (a) The seven different assessment text sections. From the assessment text sections, one of the sections is Conservation actions, not to be confused with the topic named Conservation actions. (b) IUCN Red List categories. (c) Systems inhabited by species. Freshwater refers inland freshwater. All three refers to species that inhabit all three systems together. (d) Taxonomic class incorporating the top 25\% of all classes covered by the IUCN. This top 25\% includes only taxonomic classes with more than 325 species within.}
\label{fig:topic heatmap}
\end{figure}
\FloatBarrier

Patterns were uncovered by grouping the topic probabilities according to four diffent groups (\Cref{fig:topic heatmap}). The different AST sections feature topics that relate to what is described within the sections (\Cref{fig:topic heatmap}(a)). There are a few prominent relationships between the text sections and themes of topics, such as the theme of Ecology for the Habitat section; theme of Threats for the Threats section; theme of Conservation Actions for the Conservation Actions section.

Three topics ("Habitat loss", "Terrestrial ecology" and "Spatial data") stand out across all the categories the when the topic probabilities were grouped by Red List Categories (\Cref{fig:topic heatmap}(b)). As for the Systems group, topics "Forest loss", "Habitat loss" and "Range, occurrence and occupancy" stand out.

The topic "Habitat loss" stand out across all Systems categories and Taxonomic classes (\Cref{fig:topic heatmap}(c) and \Cref{fig:topic heatmap}(d)). The topic "Marine population structure" stands out for \textit{Anthozoans}. The topic "Fisheries threats" stand out for \textit{Chondrichthyes}. The topics "Aquatic ecology" and "Terrestrial ecology" stand out for aquatic and terrestrial species respectively.

\subsection{Topic distribution across the world}

The following figures are visual representations of the proportion of number of species AST that contain a topic at each pixel to the total number of species present at the same pixel. Note that any color that is not ivory on the map represents more than half of the AST of species present at the pixel contains the topic. Due to the resolution of the map and the breadth of information enclosed, not all of the ratios can be properly represented.

% put subfigure titles on top
\begin{figure}[!htbp]
    \centering
        \begin{subfigure}[t]{0.49\textwidth}
        \centering
        \caption{Forest loss}
        \includesvg[width=\linewidth]{../Data/mammals_range/rasters/Forest_loss.svg}
        \label{subfig:for los}
        \end{subfigure}
        %
        \begin{subfigure}[t]{0.49\textwidth}
        \centering
        \caption{Water pollution threat}
        \includesvg[width=\linewidth]{../Data/mammals_range/rasters/Water_pollution_threat.svg}
        \label{subfig:wat pol}
        \end{subfigure}
        %
        \begin{subfigure}[t]{0.49\textwidth}
        \centering
        \caption{Habitat loss}
        \includesvg[width=\linewidth]{../Data/mammals_range/rasters/Habitat_loss.svg}
        \label{subfig:hab los}
        \end{subfigure}
        %
        \begin{subfigure}[t]{0.49\textwidth}
        \centering
        \caption{Forest fragmentation}
        \includesvg[width=\linewidth]{../Data/mammals_range/rasters/Forest_fragmentation.svg}
        \label{subfig:for fra}
        \end{subfigure}
    \caption{Proportion of mammalian species expressing the given topics (a) "Forest loss" (b) "Water pollution threat" (c) "Habitat loss" and (d) "Forest fragmentation" each pixel for this group of topics that are in the Habitat loss theme}
    \label{fig:hab los theme}
\end{figure}
\FloatBarrier

The four topics in the Habitat loss theme appear across the four heatmaps in a complementary pattern when visualised (\Cref{fig:hab los theme}). The "Forest loss" topic stands out only for terrestrial regions. The topics "Habitat loss" and "Forest fragmentation" both stand out mostly for terrestrial regions, and also for aquatic regions in the southern temperate zone. "Water pollution threat" appears globally, but appears in less than half of the mammalian species present in Greenland. All four topics appear in high proportions, more than 0.85, in Australia and New Zealand.

\begin{figure}[!htbp]
    \centering
        \begin{subfigure}[t]{0.49\textwidth}
        \centering
        \caption{Population dynamics}
        \includesvg[width=\linewidth]{../Data/mammals_range/rasters/Population_dynamics.svg}
        \label{subfig:pop dyn}
        \end{subfigure}
        %
        \begin{subfigure}[t]{0.49\textwidth}
        \centering
        \caption{Marine population structure}
        \includesvg[width=\linewidth]{../Data/mammals_range/rasters/Marine_population_structure.svg}
        \label{subfig:mar pop}
        \end{subfigure}
        \begin{subfigure}[t]{0.49\textwidth}
        \centering
        \caption{Spatial data}
        \includesvg[width=\linewidth]{../Data/mammals_range/rasters/Spatial_data.svg}
        \label{subfig:spa dat}
        \end{subfigure}
        %
        \begin{subfigure}[t]{0.49\textwidth}
        \centering
        \caption{Range, occurrence and occupancy}
        \includesvg[width=\linewidth]{../Data/mammals_range/rasters/Range_occurrence_and_occupancy.svg}
        \label{subfig:ran occ occ}
        \end{subfigure}
    \caption{Proportion of mammalian species expressing the given topics (a) "Population dynamics" (b) "Marine population structure" (c) "Spatial data" and (d) "Range, occurrence and occupancy" at each pixel. (a) and (b) belong to the Population theme while (c) and (d) belong to the Range theme.}
    \label{fig:pop ran theme}
\end{figure}
\FloatBarrier

The two population topics show opposite patterns, "Population dynamics" appear alomost exclusively in Terrestrial regions while "Marine population structure" is more prominent in aquatic regions (\Cref{subfig:pop dyn} and \Cref{subfig:mar pop}). The "Spatial data" topic also stands out globablly at more than 0.50, while the "Range, occurrence and occupancy" topic stands out in Australia, New Zealand, Chile and the coast of the Mediterranean sea.

\begin{figure}[!htbp]
    \centering
        \begin{subfigure}[t]{0.49\textwidth}
        \centering
        \caption{Terrestrial ecology}
        \includesvg[width=\linewidth]{../Data/mammals_range/rasters/Terrestrial_ecology.svg}
        \label{subfig:ter eco}
        \end{subfigure}
        %
        \begin{subfigure}[t]{0.49\textwidth}
        \centering
        \caption{Aquatic ecology}
        \includesvg[width=\linewidth]{../Data/mammals_range/rasters/Aquatic_ecology.svg}
        \label{subfig:aqu eco}
        \end{subfigure}
        %
        \begin{subfigure}[t]{0.49\textwidth}
        \centering
        \caption{Fisheries threats}
        \includesvg[width=\linewidth]{../Data/mammals_range/rasters/Fisheries_threats.svg}
        \label{subfig:fis thr}
        \end{subfigure}
        \begin{subfigure}[t]{0.49\textwidth}
        \centering
        \caption{Conservation actions}
        \includesvg[width=\linewidth]{../Data/mammals_range/rasters/Conservation_actions.svg}
        \label{subfig:con act}
        \end{subfigure}
        %
        \begin{subfigure}[t]{0.49\textwidth}
        \centering
        \caption{Area based protection}
        \includesvg[width=\linewidth]{../Data/mammals_range/rasters/Area_based_protection.svg}
        \label{subfig:are bas pro}
        \end{subfigure}
        %
        \begin{subfigure}[t]{0.49\textwidth}
        \centering
        \caption{Monitoring and rewilding}
        \includesvg[width=\linewidth]{../Data/mammals_range/rasters/Monitoring_and_rewilding.svg}
        \label{subfig:mon rew}
        \end{subfigure}
    \caption{Proportion of mammalian species expressing the given topics (a) "Terrestrial ecology" (b) "Aquatic ecology" (c) "Fisheries threats" (d) "Conservation actions" (e) "Area based protection" and (f) "Monitoring and rewilding at each pixel. (a), (b) and (c) belong to the Ecology theme while (d), (e) and (f) belong to the Conservation actions theme.}
    \label{fig:eco con act theme}
\end{figure}
\FloatBarrier

The two ecology topics show opposite patterns (\Cref{subfig:ter eco} and \Cref{subfig:aqu eco}). "Terrestrial ecology" appears almost exclusively on terrestrial regions. While "Aquatic ecology" is more than 0.50 globally, it is most prominent (more than 0.95) in the aquatic regions. The topic "Fisheries threats" is also most prominent in aquatic regions (more than 0.95) (\Cref{subfig:fis thr}).

The topics "Conservation actions" and "Monitoring and rewilding" stands out globally (\Cref{subfig:con act} and \Cref{subfig:are bas pro}). The "Area based protection" topic stands out in South America, Africa, New Zealand and Australia (\Cref{subfig:mon rew}).
\pagebreak

\section{Discussion}
Discuss results
\pagebreak

\section{Conclusion}
Make conclusion
\pagebreak

\section{References}
\printbibliography[heading=none]
\pagebreak

\pagenumbering{roman}

\section{Appendix}
testing one two three

\end{document}   