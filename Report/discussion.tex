% partition into subsections:

In this paper, I explored the application of rapidly summarising topics on the IUCN AST. My results suggest that this was successful in discovering themes from the texts and help better understand relationships within the AST as a whole, without the need to go through every individual assessment. This method could be used alongside meta-analyses on the IUCN's data in the future to generate additional insights.

% (explaining and writing down all of the thought process I go through)

\subsection{Latent topics in the supporting texts}

The latent topics in the supporting texts revealed by the model demonstrates some fundamental use of language that can even be detected by machines. The intention of the species AST summaries is apparent when considering the topics featured in the Rationale section of the Red List assessments (\Cref{fig:topic heatmap}). Red List categories are given based largely on the population and range \parencite{iucn2012}, this relates to how the information are selected and shown on the summaries. The summaries let readers promptly find out the rationale behind the species' given Red List category. The description of the Habitat section (\Cref{tab:example and desc of data}) corresponds directly to the names of the two ecology topics "Terrestrial ecology" and "Aquatic ecology", which are also the top two topics in the section. Habitat loss is the main driver of biodiversity loss \parencite{diaz2019pervasive, pimm2014biodiversity}, thus would be mentioned frequently in the Habitat section for species AST, which was detected by the model and revealed through the top three topics for the Threats section (\Cref{fig:topic heatmap}). As for the Population section, the two of the top three topics (\Cref{fig:topic heatmap}) describe population, and the top topic is "Common threats". The description of the Population section (\Cref{tab:example and desc of data}) shows that the population trend is described, which could be a reason why "Common threats" is the top topic as threats are intertwined with population trends. However, "Common threats" could also be mislabeled as the tokens within the topic (\Cref{tab:topic label and top 16 words}) lean toward population rather than threats, despite the second top token being "threat". The top two topics in Range section also properly describe the section, being the only two topics that are about range. The top two topics in the Use and trade section brings an interesting point to attention. When thinking about the two topics "Habitat loss" and "Fisheries threats" and the phrase "Use and trade", one would think of fisheries threats being closer related to the phrase compared to habitat loss. However, the use and trade of plant species are logically contributing to habitat loss as the plants themselves could play the role of a habitat for other species. Finally, in the Conservation actions section, the top two topics are conservation related and have been detected by the model.

\subsection{Latent topics in the Red List categories, systems and taxonomic classes}

Across Red List categories, the probability of the "Spatial data" topic increases down the categories. This shows that spatial data of species are increasingly mentioned in the AST as the category assigned increases in severity, all the way until the species is categorised as extinct. This relates back to how the species are categorised, where the spatial data of the species play a large role \parencite{iucn2012}. The probability of "Habitat loss" being almost uniform across all categories also relates back to how habitat loss the main driver of biodiversity loss \parencite{diaz2019pervasive, pimm2014biodiversity}.

AST of species in the \textit{Anthozoa} (includes sea anemones, and corals) class prominently featuring "Marine population structure" is presumably related to the topic containing the tokens "coral" and "reef" (\Cref{tab:topic label and top 16 words}). Beyond that, the discussion of population structure of Anthozoans is doubtlessly due to the destruction of coral reefs all over the world \parencite{Johnson2020CoralRD,hoegh2007coral,hughes2007phase,hughes2003climate}.

The topic "Fisheries threats" standing out for \textit{Chondrichthyes}, cartilaginous fishes, is consistent with their population decline \parencite{pacoureau2021half}, and studies into use and trade for them \parencite{stevens2000effects}. Sharks are not only exploited for their fins, squalene or squalane is an oil made from Shark's liver and used in a wide range of cosmetic products, lubricants and cleaning agents \parencite{naziri2011squalene,vannuccini1999shark,gopakumar1986squalene}. Shark cartilage is a popular dietary supplement despite lacking in scientific evidence \parencite{utsunomiya2003shark}. Shark and ray meat are marketed under many different names, Flake, Huss (served in fish and chips stores in the UK), Rock Salmon, Sea Ham, and many more \parencite{vannuccini1999shark}. Contrary to popular belief, the market for sharks trade is not just limited to sharks fin in Asia, the European Union is responsible for more than twenty percent of global trade in shark meat \parencite{wwf2021shark}.

\subsection{Latent topics on mammalian species range}
By mapping the topic presence and absence onto mammalian species range data, an overview of where the topics are geographically prominent can be observed (\Crefrange{fig:hab los theme}{fig:eco con act theme}).

One would expect the topic "Forest loss" to appear much stronger in the Amazon rainforest, as there is constant deforestation happening in the Amazon \parencite{Ellwanger2020BeyondDL,Escobar2020defor}. However, this is not reflected in on the map in my results (\Cref{subfig:for los}). First, there are range data of about 160 mammals that range over the Amazon rainforest, and about 100 of them have their AST contain the "Forest loss" topic which results in a proportion of about 0.6. Second, there are about 45500 species living in the Amazon \parencite{da2005fate}, which means that my result is only representative of 0.003\% of the species that live in the Amazon. This shows that "Habitat loss" could reflect much stronger in the Amazon if range data of other species were incorporated. 

The "Water pollution threat" topic is prominent globally and reflects very strongly in developed parts of the world (North America, Europe, Australia and New Zealand) (\Cref{subfig:wat pol}). Additionally, the topic is also quite high in proportion, at more than 0.75, in developing regions (Africa and China). This shows a strong relationship between this topic and freshwater (inland waters), and anthropogenic activity. Development comes at a cost, and anthropogenic activity contributes to pollution, not just of water \parencite{zhou2020e00925,emmanuel2016new} but the global environment as a whole \parencite{Brusseau2019TheEO}. Although water pollution is also heavy in the seas \parencite{Barrett2020MicroplasticPI,Zhang2020MicroplasticPI,Eriksen2014PlasticPI}, the relatively weaker reflection from my results could again be drawn back to the lack of non mammalian species range data incorporated in the analysis.

The "Habitat loss" topic reflecting strongly on all terrestrial regions puts additional emphasis on habitat loss being the main driver of biodiversity loss \parencite{diaz2019pervasive, pimm2014biodiversity}. The topic "Forest fragmentation" reflecting strongly in South America shows that the use of language could be affecting how the topics are detected in the AST. This relates to the lower presence of "Forest loss" in the Amazon, as previously discussed, which could be a subtle difference in the language use and reveals that forest fragmentation is the main issue rather than forest loss in the Amazon. One might think that fragmentation and loss describe the same issue but they are in fact different, however fragmentation can be a subset of loss \parencite{liu2016socioeconomic,fahrig2003effects}.

The high proportion of topics "Marine population structure" and "Spatial data" globally (\Cref{subfig:mar pop} and \Cref{subfig:spa dat}) shows that these two topics underpin many of the mammalian AST, and again shows how population structure and spatial data are main factors contributing to a species' Red List category. This points to the language used in AST of marine mammals as they were detected by the model and grouped together with words that are used often in the AST of marine species. The lower proportion for topics "Population dynamics" and "Range, occurrence and occupancy" (\Cref{subfig:pop dyn} and \Cref{subfig:ran occ occ}) compared to the previous two, show that the model has also picked up the order of Red List criteria, where the amongst the criteria, population size comes before range \parencite{iucn2012}.

The topics "Terrestrial ecology" and "Aquatic ecology" (\Cref{subfig:ter eco} and \Cref{subfig:aqu eco}) also reflects the performance of the model, where it was able to distinguish the choice of words used between the AST of terrestrial and aquatic mammals. The topics "Conservation actions" and "Monitoring and rewilding" standing out globally (\Cref{subfig:con act} and \Cref{subfig:mon rew}) shows that almost all of the AST of these mammals mention these topics. This reminds us of how much biodiversity is under threat \parencite{butchart2010global}, where almost all of these mammals require conservation actions. The "Monitoring and rewilding" topic also reminds us how much the range of these mammals have been affected and that one of the main actions proposed in the AST is to continue to monitor their numbers and rewild whenever possible \parencite{thakur2020climate,perino2019rewilding,corlett2016restoration}. The topic "Area based protection" showing up mostly on terrestrial regions (\Cref{subfig:are bas pro}) shows that area based protection is mentioned mostly in terrestrial mammals. Area protection is a classification of conservation measure by the IUCN Red List \parencite{dudley2008guidelines} which involves setting up or expanding areas such as national parks, nature reserves, marine protected areas and others that would contribute to species conservation. The topic not showing up in the oceans could also be related to the small number of oceanic marine mammals. Additionally, there are no barriers restricting the movement of marine mammals and they are often migratory which makes it difficult in setting up protected areas for them \parencite{Lascelles2014MigratoryMS,Freeman2003DISTRIBUTIONOH}. There is also the difficulty of setting up area protection in the oceans, where protecting areas in the oceans would require vast amounts of resources. Monitoring the areas that are far away from land is difficult. Defense and deterence for the areas would also be required, and setting up strategies for the areas would be counterproductive as that would contribute to disturbance of the local biodiversity.

\subsection{Limitations}
Marine mammals are not representative of oceanic species, if other range data were available that would better represent topic presence in the oceanic regions. However, the concepts presented can be extrapolated to hypothesise the patterns for others species, where topic prominence plotted spatially would not be limited to mammals within the respective space. It would be likely that topics such as "Spatial data", "Conservation actions" and "Water pollution threat" reflect just as or even stronger if the other species were included in the analysis.

Despite my goal to discover latent topics, the topics that were discovered by the models do not seem to be latent. This could result from the thresholds that I had set for the dictionary, where tokens appearing in more than 50\% of documents were removed being too high. Future work can try a lower threshold if they intend to look into latent themes. However, it could also result from the language use in the AST, where it follows a strict set of guidelines thus making latent themes much harder to discover.