The findings of this study can be the foundation for prospective research. First, the inclusion of range data from other species could improve the resolution of the maps of topic presence in proportion to species present. The amount of information revealed from mapping topics to the range data of only 11842 mammalian species is already informative, hopefully a bigger picture can be put together by expanding the maps to include the range data of other species available from the IUCN. Second, a separate model could be trained with the intention of comparing the resulting topic distribution of the Rationale section with the topic distribution of the full AST to measure how well the summaries capture each of the AST. Additionally, as I proposed in the introduction, these results can be used in conjunction with meta analyses conducted with the IUCN's data to gain more meaningful insights, to better influence conservation efforts. I hope that my study sets a precedence for further application of natural language processing methods in the Red List AST. Along with how fast the field of natural language processing is advancing, I believe there are more to uncover in this large trove of plain text information, either through the use of state of the art models and/or through another pair of curious and creative lenses. 